\documentclass[english]{article}
\title{MiniPascal compiler documentation}
\usepackage[utf8]{inputenc}
\usepackage{babel}
\usepackage{amsmath}
\usepackage{hyperref}
\usepackage{pdfpages}
\author{Timo Mäki
\\Code generation (spring 2016)
\\MiniPascal compiler}
\date{\today}

\begin{document}
\maketitle

\section{Architecture}
The program is written in C\#6, and a .NET 4.6.1 compatible environment is expected for the execution of the compiler, and the resulting CIL bytecode.
The program is build using Visual Studio 2015, and executed on command line, eg. "MiniPascal.exe myProgram.txt".
The compiler includes a test suite that uses NUnit test framework.
The tests are run using NUnit3 test adapter for Visual Studio 2015.
The test adapter must downloaded to visual studio as an extension.
The NUnit framework can be downloaded through Nuget with package restoring.
The tests builds an executable, which output is read by the tests.
Note that the tests may fail sometimes, if the process read access is not released in time.
A rerun fixes the problem.
Tests also do not test all static analysis errors, though most tests still fail with incorrect input.

The lexer is defined by three types, \emph{Scanner}, \emph{TokenConstruction}, \emph{IScannerState}.
The Scanner keeps reading the input, the scanner state is moved according to input and the TokenConstruction holds the build tokens.
The scanner states form a finite state machine.

The input stream for the scanner is called \emph{SourceStream}.
Its backing storage differs when run with the commandline and when run with the test runner.

The parser is defined by a partial class, composed of three files.
\emph{ParserStatements} contains grammar rules for statements, \emph{ParserExpressions} contains grammar rules for expressions.
\emph{Parser} contains the rest, like types, blocks and helper methods.

Most grammar rules contain their own class for results.
For example, there is a class for If statement, while etc.

Most of the operators (and .size) are their own classes, and are attached to their \emph{SimpleType}.
\emph{MiniPascalType} contains the SimpleType and a flag whether the type is array.

Semantic analysis involves the \emph{Scope} class.
It contains the identifiers of the current, and the previous, scope.
The analysis is started by a check of indentifiers, that they are declared and unique in the scope.
Second it performs the type check of the operations.
No operation is allowed for two different types, although certain operations may return a different type than the original types.
The scope class is discarded after the analysis.

Like semantic analysis, code generation is performed in the statement and expression classes themselves.
The \emph{CILEmitter} class holds the necessary Reflection.Emit types for generation.
For example, it stores the variable and procedure names.
In general, there is no need to import Reflection namespace outside this class.

\section{Known missing featurs}
The compiler aborts when it counters an error.
The implementation does not support using the predefined indentifiers as indentifiers.

%\section{Architectural issues}
%The CILEmitter class is not well designed for object oriented programming.
%Many of its methods are called only once, and they should be implemented instead by %the caller.
%Array size should not be attached to array type, instead there should be a specific %array declaration class.

\section{Language implementation-level decisions}
For all programs, the limitations of the Common Language Runtime apply.
In variable declarations, integers and reals are set to 0, booleans to false and strings to empty string literal.
Same applies for arrays, expect for strings which are set to null.
Integers and reals are 32 bit.
Booleans are internally integers.

For string comparison, the BCL String.Compare is used.
The "a" $<$ "b" comparison returns true, since a is before b in alphabetic ordering.
For Booleans the integer comparision is used.
String.Equals is used for string equality comparison.
Integers, reals and booleans uses the CIL instructions for the previous.
String.Concat for string concatenation.

All arguments of method calls are evaluated left to right.
All equal operations are also evaluated left to right (eg. in 1 + 2 + 3, the 1 + 2 is first counted).

Method writeln is a series of calls to Console.Write.
Writeln does not automatically add new line character sequence (since it is not required, and is quite easy to add).

Read is treated as a series of assigments.
It calls to Console.ReadLine, and parses the result when needed.
Real parsing uses InvarianCulture from System.Globalisation.
Parse failures raise an unhandled exception from the BLC library.

Assert is an If statement, calling Environment.Exit(-1) when failing.

Functions are static methods in the emitted code (and procedures are functions with void as return type).
All variables are allocated to the stack, and all previously allocated variables are passed by reference to the method.
A scope can have up to 256 variables at the same time.

\section{Major problems}
The largest implementation issue was the by-reference variables of methods.
Particulary, the assigments, loads and arguments had to be done differently, depending wheter the variable was local stack, local parameter, array, reference parameter or reference scope.
The solution was to ensure all possible variables are always available.
Variables are accessed through a series of if-else statements (like in Assigment statement or variable operand).

\section{Semantic analysis}
\begin{itemize}
\item Assigment type
\item Operator type
\item Operator exists for type
\item Declared identifier
\item Unique identifier
\item Array index integer type
\item While/If expression boolean type
\item Method call parameter count match
\item Reference only for variables
\item Size for arrays only
\end{itemize}

Notably, there is no check that return is called, or that its type is correct.

\section{Modified grammar}
\begin{enumerate}
\item Simple statement should lead into new reference start rule, which would then split to call and assigment, since both of them can start with an identifier.
\item Factor should contain same kind of reference start rule as previously, for call and variable.
\item If statement should contain only: "if" $<$Boolean expr$>$ "then" $<$statement$>$ [ "else" $<$statement$>$ ].
\item In factor, size should be added to the end of every statement, or create a new rule where first a factor is read, and optionally size.
\end{enumerate}

\section{Abstract Syntax Tree}
\begin{enumerate}
\item AbtractSyntaxTree has ScopedProgram
\item ScopedProgram has IStatements
\item IStatement is implemented by Assert, Assigment, Read, Declaration, If, NoReturn Call, Print, Procedure, Return, While
\begin{enumerate}
\item Assert has IExpression
\item Assigment has an IExpression and identifier
\item Declaration has identifiers and their type
\item If has IExpression and two IStatements
\item NoReturnCall has a Call
\item Print has IExpressions
\item Procedure has a ScopedProgram, MiniPascalType, identifier and variables with their identifiers, types and reference flag
\item Read has assigment statements
\item Return has an IExpression
\item While has an IExpression and a ScopedProgram
\end{enumerate}
\item IExpression is implemented by Expression and SignedExpression
\begin{enumerate}
\item Expression has IExpressions or IOperands and their operators
\item SignedExpression is Expression and has a sign
\end{enumerate}
\item IOperand is implemented by Voolean, Integer, String and Real literals, Call, ExpressionOperand, Size, Unary, VariableOperand and Reading
\begin{enumerate}
\item Call has an identifier and IExpressions
\item Reading has variable and its type
\item ExpressionOperand has IExpression
\item Size has and IOperand
\item Unary has an IOperand and operator
\item VariableOperand has identifier and its type
\end{enumerate}


\end{enumerate}



\includepdf[pages={-}]{../Code_Gen_Project_2016_Spring.pdf}
\includepdf[pages={-}]{../minipascalsyntax.pdf}

\end{document}
